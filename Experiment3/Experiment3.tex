\documentclass{article}
\usepackage{color}
\usepackage[UTF8]{ctex}
\usepackage{graphicx}
\usepackage{float}

\begin{document}
    \title{第三次实验}
    \author{汤琦 \\ 23020007110\\ https://github.com/wlmrh/System-development-tool-basics}
    \date{\today}
    \maketitle

    \pagenumbering{arabic}
    \tableofcontents
    \newpage

    \setlength{\parindent}{0pt}
    \setcounter{page}{1}
    \sloppy
\section{Python基础}
    \subsection{Q1}
    有四个数字:1、2、3、4,能组成多少个互不相同且无重复数字的三位数?各是多少?
    \begin{verbatim}
        for i in range(1, 5):
            for j in  range (1, 5):
                for k in range(1, 5):
                    if i != j and j != k and i != k:
                        print(i, j, k)
    \end{verbatim}

    \subsection{Q2}
    一个整数,它加上100后是一个完全平方数,再加上168又是一个完全平方数,请问该数是多少?
        \begin{verbatim}
            sub = 168
            for i in range(90):
                for j in range (1, i):
                    if (i + j) * (i - j) == sub:
                        print(i ** 2 - 268)
                        break            
        \end{verbatim}
    
    \subsection{Q3}
    输入三个整数x,y,z,请把这三个数由小到大输出。
        \begin{verbatim}
            arr = []
            for i in range(3):
                arr.append(int(input()))
            arr.sort()
            for i in range(3):
                print(arr[i])
        \end{verbatim}
    \subsection{Q4}
    斐波那契数列
    \begin{verbatim}
        a = 0
        b = 1
        final = int(input())
        for i in range(final):
            print(b)
            a, b = b, a + b
    \end{verbatim}

    \subsection{Q5}
    将一个列表的数据复制到另一个列表中。
    \begin{verbatim}
        arr1 = []
        arr2 = []
        len = int(input())
        for i in range(len):
            arr1.append(int(input()))
        arr2 = arr1[:]
        print('arr2: ')
        for i in range(len):
            print(arr2[i])
    \end{verbatim}
    
    \subsection{Q6}
    输出 9*9 乘法口诀表
    \begin{verbatim}
        for i in range(1, 10):
            for j in range(1, i + 1):
                print('%d x %d = %d' % (i, j, i * j), end = " ")
            print('\n')
    \end{verbatim}

    \subsection{Q7}
    暂停一秒输出
    \begin{verbatim}
        import time
            out = ['adsfa', 'fwdcsc']
            print(out[0])
            time.sleep(3)
            print(out[1])
    \end{verbatim}

    \subsection{Q8}
    有一对兔子,从出生后第3个月起每个月都生一对兔子,小兔子长到第三个月后每个月又生一对兔子,假如兔子都不死,问每个月的兔子总数为多少?
    \begin{verbatim}
        a, b = 1, 1
            len = int(input())
            for i in range(len):
                a, b = b, a + b
            print(b)
    \end{verbatim}

    \subsection{Q9}
    判断101-200之间有多少个素数,并输出所有素数。
    \begin{verbatim}
        import math
        cnt = 0
        final = []
        for i in range(101, 201):
            good = 0
            for j in range(2, int(math.sqrt(i)) + 1):
                if i % j == 0:
                    good = 1
                    break
            if good == 0:
                cnt += 1
                final.append(i)

        print('一共有%d个素数' % cnt)
        for i in final:
            print(i, end = " ")
    \end{verbatim}

    \subsection{Q10}
    打印出所有的"水仙花数",所谓"水仙花数"是指一个三位数,其各位数字立方和等于该数本身。例如:153是一个"水仙花数",因为153=1的三次方+5的三次方+3的三次方。
    \begin{verbatim}
        for i in range(1, 10):
            for j in range(10):
                for k in range(10):
                    if i * 100 + j * 10 + k == i ** 3 + j ** 3 + k ** 3:
                        print(i * 100 + j * 10 + k)
    \end{verbatim}

    \subsection{Q11}
    将一个正整数分解质因数。例如:输入90,打印出90=2*3*3*5
    \begin{verbatim}
        import math

        solve = {}
        num = int(input())
        origin = num
        for i in range(2, int(math.sqrt(num)) + 1):
            if num == 1:
                break
            if num % i == 0:
                solve[i] = 1
                num /= i
            while num % i == 0:
                solve[i] += 1
                num /= i

        if num != 1:
            solve[num] = 1

        print('%d =' % origin, end = ' ')
        first = 0
        for a, b in solve.items():
            for i in range(b):
                if first != 0:
                    print('x %d' % a, end = ' ')
                else:
                    print('%d' % a, end = ' ')
                    first = 1
    \end{verbatim}

    \subsection{Q12}
    学习成绩>=90分的同学用A表示,60-89分之间的用B表示,60分以下的用C表示。
        \begin{verbatim}
            score = int(input('请输入成绩: '))

            if score >= 90:
                print('A')
            elif score >= 60:
                print('B')
            else:
                print('C')
        \end{verbatim}

    \subsection{Q13}
    求s=a+aa+aaa+aaaa+aa...a的值,其中a是一个数字。例如2+22+222+2222+22222(此时共有5个数相加),几个数相加由键盘控制。
    \begin{verbatim}
        n = int(input('请输入n的值: '))
        a = int(input('请输入a的值: '))
        
        ans = 0
        
        for i in range(n):
            ans += (10 ** i) * a * (n - i)
        
        print(ans)
    \end{verbatim}

    \subsection{Q14}
    一个数如果恰好等于它的因子之和,这个数就称为"完数"。例如6=1+2+3.编程找出1000以内的所有完数。
    \begin{verbatim}
        import math

        ans = []
        for i in range(1, 1001):
            rcd = []
            for j in range(1, i):
                if i % j == 0:
                    rcd.append(j)
        
            if sum(rcd) == i:
                ans.append(i)
        
        for i in ans:
            print(i)
    \end{verbatim}
\section{Python视觉编程}
    \subsection{Q1}
    导入图片并展示
    \begin{verbatim}
        from PIL import Image
        from pylab import *

        axis('off')
        pil_im = Image.open("test.jpg")
        im = array(pil_im)
        imshow(im);
    \end{verbatim}

    \subsection{Q2}
    调整图片灰度并展示
    \begin{verbatim}
        from PIL import Image
        from pylab import *

        axis('off')
        pil_im = Image.open("test.jpg")
        grey = pil_im.convert("L")
        im = array(grey)
        imshow(im)
        axis('off')
    \end{verbatim}

    \subsection{Q3}
    简单灰度变换并输出
    \begin{verbatim}
        from PIL import Image
        from pylab import *

        axis('off')
        pil_im = Image.open("test.jpg")
        im = array(pil_im)
        pic = 255.0 * (im/255.0)**2
        imshow(pic)
        axis('off')
    \end{verbatim}

    \subsection{Q4}
    图片复制反转
        \begin{verbatim}
            from PIL import Image
            from pylab import *

            axis('off')
            pil_im = Image.open("test.jpg")
            region = (1000, 0, 3500, 2000)
            region_rev = pil_im.crop(region)
            region_rev = region_rev.transpose(Image.ROTATE_180)
            imshow(array(region_rev))
            axis('off')
        \end{verbatim}

        \subsection{Q5}
        显示图像轮廓
        \begin{verbatim}
            from PIL import Image
            from pylab import *

            axis('off')
            pil_im = Image.open("test.jpg")
            im = array(pil_im)
            figure()
            gray()
            contour(im, origin='image')
            axis('equal')
            axis('off')
        \end{verbatim}

        \subsection{Q6}
        图像模糊化
            \begin{verbatim}
                from numpy import *
                from scipy.ndimage import filters
                from PIL import Image
                from pylab import *

                axis('off')
                pil_im = Image.open("test.jpg")
                im = array(pil_im)
                for i in range(3):
                    im[:, :, i] = filters.gaussian_filter(im[:, :, i], 10)
                imshow(im)
                axis('off')
            \end{verbatim}

\end{document}